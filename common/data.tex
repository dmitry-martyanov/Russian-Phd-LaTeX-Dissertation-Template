%%% Основные сведения %%%
\newcommand{\thesisAuthorLastName}{Мартьянов}
\newcommand{\thesisAuthorOtherNames}{Дмитрий Сергеевич}
\newcommand{\thesisAuthorInitials}{Д.\,С.}
\newcommand{\thesisAuthor}             % Диссертация, ФИО автора
{%
    \texorpdfstring{% \texorpdfstring takes two arguments and uses the first for (La)TeX and the second for pdf
        \thesisAuthorLastName~\thesisAuthorOtherNames% так будет отображаться на титульном листе или в тексте, где будет использоваться переменная
    }{%
        \thesisAuthorLastName, \thesisAuthorOtherNames% эта запись для свойств pdf-файла. В таком виде, если pdf будет обработан программами для сбора библиографических сведений, будет правильно представлена фамилия.
    }
}
\newcommand{\thesisAuthorShort}        % Диссертация, ФИО автора инициалами
{\thesisAuthorInitials~\thesisAuthorLastName}
\newcommand{\thesisUdk}                % Диссертация, УДК
{539.142.3, 539.172}
\newcommand{\thesisTitle}              % Диссертация, название
{Оптическая модель для учета колебательных и вращательных возбуждений при рассеянии нуклонов на тяжелых ядрах}
\newcommand{\thesisSpecialtyNumber}    % Диссертация, специальность, номер
{01.04.16}
\newcommand{\thesisSpecialtyTitle}     % Диссертация, специальность, название
{физика атомного ядра и элементарных частиц}
\newcommand{\thesisDegree}             % Диссертация, ученая степень
{кандидата физико-математических наук}
\newcommand{\thesisDegreeShort}        % Диссертация, ученая степень, краткая запись
{канд. физ.-мат. наук}
\newcommand{\thesisCity}               % Диссертация, город написания диссертации
{Минск}
\newcommand{\thesisYear}               % Диссертация, год написания диссертации
{2021}
\newcommand{\thesisOrganization}       % Диссертация, организация
{Национальная академия наук Беларуси \\ ГОСУДАРСТВЕННОЕ НАУЧНОЕ УЧРЕЖДЕНИЕ \\ <<ОБЪЕДИНЕННЫЙ ИНСТИТУТ ЭНЕРГЕТИЧЕСКИХ И ЯДЕРНЫХ ИССЛЕДОВАНИЙ~--- СОСНЫ>>}
\newcommand{\thesisOrganizationShort}  % Диссертация, краткое название организации для доклада
{ГНУ <<ОИЭЯИ~--- Сосны>>}

\newcommand{\thesisInOrganization}     % Диссертация, организация в предложном падеже: Работа выполнена в ...
{Государственном научном учреждении <<Объединенный институт энергетических и ядерных исследований~--- Сосны>> Национальной академии наук Беларуси}

\newcommand{\supervisorFio}            % Научный руководитель, ФИО
{Суховицкий Ефрем Шоломович}
\newcommand{\supervisorRegalia}        % Научный руководитель, регалии
{кандидат физико-математических наук,\par заведующий лабораторией}
\newcommand{\supervisorFioShort}       % Научный руководитель, ФИО
{Е.\,Ш.~Суховицкий}
\newcommand{\supervisorRegaliaShort}   % Научный руководитель, регалии
{канд. физ.-мат. наук, зав. лаб.}

 \newcommand{\supervisorDead}{}
%% \newcommand{\supervisorTwoDead}{}        % Рисовать рамку вокруг фамилии
%% \newcommand{\supervisorTwoFio}           % Второй научный руководитель, ФИО
%% {\fixme{Фамилия Имя Отчество}}
%% \newcommand{\supervisorTwoRegalia}       % Второй научный руководитель, регалии
%% {\fixme{уч. степень, уч. звание}}
%% \newcommand{\supervisorTwoFioShort}      % Второй научный руководитель, ФИО
%% {\fixme{И.\,О.~Фамилия}}
%% \newcommand{\supervisorTwoRegaliaShort}  % Второй научный руководитель, регалии
%% {\fixme{уч.~ст.,~уч.~зв.}}

\newcommand{\opponentOneFio}           % Оппонент 1, ФИО
{Левчук Михаил Иванович}
\newcommand{\opponentOneRegalia}       % Оппонент 1, регалии
{доктор физико-математических наук}
\newcommand{\opponentOneJobPlace}      % Оппонент 1, место работы
{Государственное научное учреждение <<Институт физики имени Б.\,И.\,Степанова>> Национальной академии наук Беларуси, лаборатория теоретической физики}
\newcommand{\opponentOneJobPost}       % Оппонент 1, должность
{главный научный сотрудник}


\newcommand{\opponentTwoFio}           % Оппонент 2, ФИО
{\fixme{Фамилия Имя Отчество}}
\newcommand{\opponentTwoRegalia}       % Оппонент 2, регалии
{\fixme{кандидат физико-математических наук}}
\newcommand{\opponentTwoJobPlace}      % Оппонент 2, место работы
{\fixme{Основное место работы c длинным длинным длинным длинным названием}}
\newcommand{\opponentTwoJobPost}       % Оппонент 2, должность
{\fixme{старший научный сотрудник}}

%% \newcommand{\opponentThreeFio}         % Оппонент 3, ФИО
%% {\fixme{Фамилия Имя Отчество}}
%% \newcommand{\opponentThreeRegalia}     % Оппонент 3, регалии
%% {\fixme{кандидат физико-математических наук}}
%% \newcommand{\opponentThreeJobPlace}    % Оппонент 3, место работы
%% {\fixme{Основное место работы c длинным длинным длинным длинным названием}}
%% \newcommand{\opponentThreeJobPost}     % Оппонент 3, должность
%% {\fixme{старший научный сотрудник}}

\newcommand{\leadingOrganizationTitle} % Ведущая организация, дополнительные строки. Удалить, чтобы не отображать в автореферате
{Учреждение образования <<Гомельский государственный технический университет имени П.\,О.\,Сухого>>}

\newcommand{\defenseDate}              % Защита, дата
{\fixme{DD mmmmmmmm YYYY~г.~в~XX часов}}
\newcommand{\defenseCouncilNumber}     % Защита, номер диссертационного совета
{Д\,01.05.02}
\newcommand{\defenseCouncilTitle}      % Защита, учреждение диссертационного совета
{Институте физики имени Б.\,И.\,Степанова Национальной академии наук Беларуси}
\newcommand{\defenseCouncilAddress}    % Защита, адрес учреждение диссертационного совета
{220072, Минск, пр.~Независимости, 68-2}
\newcommand{\defenseCouncilPhone}      % Телефон для справок
{+375~(17)~284-15-59}
\newcommand{\defenseCouncilEmail}      % Email
{vyblyi@gmail.com}

\newcommand{\defenseSecretaryFio}      % Секретарь диссертационного совета, ФИО
{Выблый\,Ю.\,П.}
\newcommand{\defenseSecretaryRegalia}  % Секретарь диссертационного совета, регалии
{кандидат физико-математических наук}            % Для сокращений есть ГОСТы, например: ГОСТ Р 7.0.12-2011 + http://base.garant.ru/179724/#block_30000

\newcommand{\synopsisLibrary}          % Автореферат, название библиотеки
{Центральной научной библиотеке имени Якуба Коласа Национальной академии наук Беларуси}
\newcommand{\synopsisDate}             % Автореферат, дата рассылки
{\fixme{DD mmmmmmmm}\the\year~года}

% To avoid conflict with beamer class use \providecommand
\providecommand{\keywords}%            % Ключевые слова для метаданных PDF диссертации и автореферата
{}
