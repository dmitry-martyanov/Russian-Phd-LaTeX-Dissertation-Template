%\chapter*{Введение}                         % Заголовок
%\addcontentsline{toc}{chapter}{Введение}    % Добавляем его в оглавление

Знание сечений взаимодействия нуклонов с тяжелыми ядрами необходимо для разработки различных реакторных установок, основанных на делении атомных ядер. Очень важны, в частности, данные по захвату и неупругому рассеянию нейтронов. В то же время текущий статус оцененных данных по неупругому рассеянию нейтронов для актинидов является неудовлетворительным \cite{INDCReport2012}: есть существенные расхождения в оценках сечений и энергетических распределений реакций (n,n’), (n,2n), (n,3n) в диапазоне энергий налетающих нейтронов от 200 кэВ до нескольких МэВ. Такое положение дел требует, в частности развития моделей, описывающих взаимодействие нуклонов с ядрами. 


Оптическая модель является одним из фундаментальных теоретических инструментов, обеспечивающих основу для оценки сечений рассеяния нуклонов на ядрах \cite{HodgsonBookRus}. Аккуратный расчет таких сечений для деформированных ядер, например актинидов, требует учета связи 

