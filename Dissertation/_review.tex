\chapter{Основные положения оптической модели и ее применение при оценке ядерных данных}

\section{Оптическая модель в современном анализе ядерных данных}

Знание сечений взаимодействия нуклонов с тяжелыми ядрами необходимо для разработки различных реакторных установок, основанных на делении атомных ядер. Очень важны, в частности, данные по захвату и неупругому рассеянию нейтронов. В то же время текущий статус оцененных данных по неупругому рассеянию нейтронов для актинидов является неудовлетворительным \cite{INDCReport2012}: есть существенные расхождения в оценках сечений и энергетических распределений реакций (n,n’), (n,2n), (n,3n) в диапазоне энергий налетающих нейтронов от 200 кэВ до нескольких МэВ. Такое положение дел требует, в частности развития моделей, описывающих взаимодействие нуклонов с ядрами. Оптическая модель является одним из фундаментальных теоретических инструментов, обеспечивающих основу для описания процессов рассеяния нуклонов на ядрах \cite{HodgsonBookRus}.

\section{Общие положения оптической модели}

Нуклон, налетающий на ядро, может вызвать несколько различных процессов. При низких энергиях налетающей частицы наиболее вероятны упругое рассеяние и поглощение с образованием составного ядра. С ростом энергии частица может захватить или выбить один или несколько нуклонов, а также оставить ядро в возбужденном состоянии при рассеянии. При этом для низких энергий сечения рассеяния и захвата набор узких часто расположенных резонансов (отвечающих состояния составного ядра), плотность которых растет с увеличением энергии налетающей частицы. При больших энергиях (сотни кэВ для актинидов) эти резонансы сливаются, а зависимость сечений от энергии становится плавной с медленными осцилляциями. 
Было найдено, что модель рассеяния на комплексном потенциале хорошо описывает осциллирующие сечения \cite{Fernbach1949, Serber1947}, а также низкоэнергетическую часть, если сечения усреднить по большому числу резонансов \cite{Feshbach1954}. Эта модель названа оптической, так как замена многочастичного нуклонного взаимодействия двухчастичным комплексным нуклон-ядерным потенциалом аналогична описанию распространения света в поглощающей и преломляющей среде с помощью показателя преломления. При этом реакции, соответствующие преломленной волне, называют прямыми, а поглощенные волны описывают образование составного ядра. Прекрасный обзор современного состояния оптической модели представлен в работе \cite{Dickhoff2019}.

\section{Подходы к решению задачи рассеяния}

Квантовомеханический анализ предполагает решение задачи рассеяния. В простейшем случае амплитуда рассеяния легко получается в рамках метода искаженных волн (борновское приближение) \cite{DavydovBookKM}, что позволяет описать множество экспериментальных данных. В то же время в ряде случаев сечения неупругого рассеяния оказываются довольно большими, указывая на то, что взаимодействия, вовлеченные в эти процессы, слишком велики и первого порядка теории возмущений оказывается недостаточно. Такая ситуация, как правило, встречается в ядрах, для которых существенную роль играют коллективные возбуждения.

\section{Метод связанных каналов}

Более аккуратное описание предлагает метод связанных каналов (обобщенная оптическая модель) \cite{Buck1963}, в рамках которого для конечного числа наиболее сильно связанных каналов рассеяния ищется точное решение уравнения Шредингера. Такой подход позволяет учесть многостадийные процессы возбуждения ядра при рассеянии. Впервые на важность метода связанных каналов при расчете рассеяния нейтронов на ядрах с выраженным коллективными свойствами указали Бор и Моттельсон \cite{Bohr1953}. Детальное изложение формализма для решения таких задач представлено в работе \cite{Tamura1965}.

\section{Развитие оптического потенциала}

Вид оптического потенциала выбирается из общих представлений о структуре ядра, нуклон-нуклонных взаимодействиях, а также исходя из имеющихся экспериментальных данных по рассеянию.
Действительную часть потенциала можно представить, как сумму взаимодействий налетающей частицы с нуклонами ядра. Из-за короткодействующего характера ядерных сил форма действительного потенциала должна быть близка к распределению ядерной плотности. Обычно радиальная зависимость выбирается в виде функции Вудса-Саксона, описывающей почти постоянную в центре и плавно спадающую у края форму.
Поглощение нуклонов ядром должно быть должно быть сосредоточено преимущественно на поверхности ядра, где плотность ядерной материи еще велика, но есть свободные состояния для налетающей частицы. При низких энергиях поглощению в центральной области ядра препятствует принцип Паули. С ростом энергии налетающей частицы поглощение появляется и в центральной области. Из-за описанных особенностей принято выделять объемное и поверхностное поглощение с радиальными зависимостями в виде функции Вудса-Саксона и ее производной соответственно, а также разными энергетическими зависимостями.
Для протонов к действительной части должно быть добавлено кулоновское отталкивание, как правило в виде потенциала однородно заряженной сферы.
Зависимость угловых распределений рассеянных частиц от их спина заставляет включать в потенциал также спин-орбитальный член, пропорциональный производной от радиальной зависимости потенциала и имеющий как действительную, так и мнимую части.
Потенциал оптической модели, полученный суммированием двухчастичных нуклон-нуклонных потенциалов, является, строго говоря, нелокальным \cite{Brueckner1095}, так как налетающая частица взаимодействует с ядерным веществом в некоторой (небольшой) окрестности. Существует несколько способов учета этого эффекта, сводящие нелокальный потенциал к эквивалентному локальному \cite{Perey1962,Lipperheide1967}, что приводит к зависимости действительной части оптического потенциала от энергии (экспоненциальный спад).
Кроме того, аккуратный анализ показывает также, что действительная часть потенциала зависит от изотопических спинов ядра и налетающей частицы \cite{LanePRL1962,LaneNP1962}. Это позволяет предсказать, как будет меняться оптический потенциал в соседних изотопах и изобарах, а также в рамках единого потенциала рассчитывать сечения прямых реакций рассеяния нейтронов, протонов и обмена зарядом (p,n) на изобар-аналоговые состояния. В более поздних работах \cite{SatchlerBook1969} была обоснована подобная зависимость и для мнимой части, однако многие авторы указывают на отсутствие зависимости мнимой части от изоспина в построенных потенциалах для различных ядер.
Попытки описать рассеяние и связанные состояние используя один и тот же ядерный потенциал привели к появлению «поляризационной» добавки к действительной части оптического потенциала, вызванной применением дисперсионного соотношения к зависящей от энергии мнимой части \cite{Lipperheide1966,Passatore1967,Mahaux1984,MahauxBook1991,Morillion2004}. В результате энергетические зависимости разных частей потенциала оказываются связаны и уменьшается количество свободных параметров оптического потенциала. Метод связанных каналов в таком анализе применялся в работе \cite{Romain1997}. Энергетические зависимости действительной и мнимой частей потенциала исследовались путем анализа данных рассеяния нуклонов при различных энергиях. В то же время удобно представить эти зависимости в функциональной форме, допускающей аналитическое вычисление дисперсионных поправок \cite{Quesada2003}. Для действительной части потенциала это уже упомянутая экспоненциальная зависимость, предложенная в работе \cite{Lipperheide1967}, интегрируемые энергетические зависимости для объемной и поверхностной частей потенциала предложены в работах \cite{Brown1961,Delaroche1989}. Удобная форма спин-орбитальной части потенциала описана в работе~\cite{Koning2003}.
Наличие энергетических зависимостей в оптическом потенциале усложняет задачу построения симметричного относительно изотопического спина потенциала. Кулоновская поправка, возникающая в дисперсионном потенциале при рассмотрении рассеяния протонов \cite{Tornow1988}, не удовлетворяет форме, предложенной Лэйном. Полностью симметричный дисперсионный потенциал, описывающий одновременно рассеяние нейтронов, протонов и реакций (p,n), удалось построить, уменьшая энергию налетающих частиц в случае протонов на величину кулоновского барьера \cite{Sun2004}.
Анализ рассеяния нуклонов при высоких энергиях (от десятков до сотен МэВ) возможен только при учете релятивистских эффектов. В рамках достаточно правдоподобных условий уравнение Дирака, описывающее рассеяние частиц с высокой энергией, оказывается эквивалентным нерелятивистскому уравнению Шредингера с кинематикой, замененной на релятивистскую, и оптическим потенциалом, умноженным на поправочный коэффициент \cite{Elton1966}. Немного иная форма такого коэффициента предложена в работе \cite{Madland1997}. В результате сохраняется единое описание процесса рассеяния в широком диапазоне энергий налетающих частиц (до 200 МэВ).

\section{Деформированный потенциал}

Многие ядра, представляющие интерес для ядерной физики, особенно ядра редкоземельных элементов и актиниды, являются значительно деформированными, что необходимо учитывать при расчетах. Прежде всего это приводит появлению низколежащих коллективных состояний ядра, которые могут быть легко возбуждены и, следовательно, сильно связаны с основным состоянием ядра. Это заставляет использовать метод связанных каналов для решения задачи рассеяния на сильно деформированных ядрах. Кроме того, существенно усложняется расчет матричных элементов перехода, а при учете деформаций в спин-орбитальной части потенциала не получается использовать быстрые численные методы (например, метод Нумерова) для решения системы связанных уравнений из-за наличия в них членов с первой производной наряду со второй. 
В типичных задачах для учета коллективных возбуждений ядро считают либо сферическим с малыми колебательными возбуждениями, либо жестким деформированным (вращательные возбуждения) \cite{Tamura1965}. С другой стороны, ядра актинидов одновременно мягкие и деформированные, однако стандартный формализм, использующий разложение оптического потенциала около сферы, обладает очень плохой сходимостью (требует много членов в разложении) из-за наличия значительной статической деформации. Это требует дальнейшего совершенствования оптической модели для ядер с колебательными и вращательными возбуждениями.

\section{Модели ядра для описания коллективного движения}

Расчет матричных элементов оптического потенциала – ключевого компонента метода связанных каналов – требует волновых функций, описывающих форму ядра для низколежащих коллективных состояний. В качестве моделей, описывающих как вращение, так и колебания ядра, можно упомянуть колебательно-вращательную модель \cite{EisenbergBookRus}, описывающую гармонические колебания аксиально-симметричного квадрупольно деформированного ядра, а также модель мягкого ротатора \cite{Porodzinskii1991}, отличающуюся учетом неаксиальной квадрупольной деформации, а также непертурбативным способом учета растяжения ядра при вращении.

\section{Существующие коды}

Одним из наиболее широко используемых кодов, осуществляющих расчеты с помощью оптической модели и метода связанных каналов, является код ECIS (Equations Couplees en Iterations Sequentielles – решение связанных уравнений методом последовательных итераций) \cite{Raynal1994}. Еще стоит выделить код FRESCO \cite{Thompson1988}, также предлагающий максимальное количество возможностей: программа общего назначения для работы по методу связанных каналов, поддержка современных моделей, открытый код, документация и продолжающаяся разработка. Можно упомянуть также несколько программ, реализующих расчеты по оптической модели, которые разработаны в различных организациях по всему миру: SCAT2000, CoH, CCOM, CHUCK и другие. В лаборатории ядерных данных Объединенного института энергетических и ядерных исследований также создан код OPTMAN \cite{OPTMANmanual2005}, реализующий функционал, аналогичный ECIS или FRESCO, и позволяющий вести дальнейшее развитие моделей. 

\section{Выводы к главе}

В данном обзоре представлены общие положения оптической модели, развитие и современное состояние потенциала модели, упомянуты программные реализации. Указано на необходимость дальнейшего развития модели для мягких деформированных ядер.
