\chapter{Развитие оптической модели для учета связи с вращательными 
уровней из полос, отличных от основной, в тяжелых деформируемых ядрах
с большой равновесной деформацией}

%\section{Основные положения оптической модели и метода связанных каналов}

%\section{Построение оптического потенциала для деформируемых ядер с большой равновесной деформацией}

\section{Параметризация формы ядра для расширенной схемы связи каналов%
\label{sec:nuclear_shape}}

Актиниды являются значительно деформированными ядрами с сильным возбуждением низколежащий коллективных уровней при неупругом рассеянии нуклонов. Используемая схема параметризации формы ядра совмещает описание так называемого жесткого ротатора с дополнительными малыми (в том числе, динамическими) квадрупольными и октупольными отклонениями от указанной формы. Жесткий ротатор здесь отвечает форме ядра со статическими аксиальными деформациями с четными мультиполями, а малые колебания описываются моделью мягкого ротатора~\cite{SRM1,SRM2}. 

%Actinides are well deformed nuclei, where low-lying collective levels are strongly excited in nucleon inelastic scattering. The employed nuclear shape parameterization combines a rigid rotor description of the ground state rotational band, with small %dynamical
%quadrupolar and octupolar vibrations around the deformed equilibrium shape. Vibrations are described in the spirit of the soft-rotator model \cite{SRM1,SRM2}, which considers nuclei to be deformable with both axial and non-axial vibrations; deformations with $\lambda \geqslant 4$ are considered axial, assuming that such deformations are usually small. Rotational bands are built on top of vibrational (single-particle) bandheads for even-even (odd) nuclei, respectively. The deformed nuclear optical potential arises from deformed instant nuclear shapes of actinide nucleus. The instant nuclear shape in a body-fixed system can be described as follows
% JMQ 14.02.2015 changed the order of terms for the more natural one. Also suppressed the comment
% explaining that spherical harmonics with \mu=0 do not depend on \theta is superfluous
% (this not Tamura's notation is just a basic property of spherical harmonics)
\begin{eqnarray}
R_i(\theta ^{\prime },\varphi ^{\prime }) &=&R_{0i}^{{}}\Bigg\{1+\underset{%
\lambda =4,6,8}{\sum }\beta _{\lambda 0}Y_{\lambda 0}(\theta ^{\prime })+
\notag \\
&&+\beta _{2}\left[ \cos \gamma Y_{20}(\theta ^{\prime })+\frac{1}{\sqrt{2}}%
\sin \gamma ~\left[ Y_{22}(\theta ^{\prime },\varphi ^{\prime
})+Y_{2-2}(\theta ^{\prime },\varphi ^{\prime })\right] \right] +
\label{SRM_R} \\
&&+\beta _{3}\left[ \cos \eta Y_{30}(\theta ^{\prime })+\frac{1}{\sqrt{2}}%
\sin \eta ~\left[ Y_{32}(\theta ^{\prime },\varphi ^{\prime
})+Y_{3-2}(\theta ^{\prime },\varphi ^{\prime })\right] \right] \Bigg\},
\notag
\end{eqnarray}%

\begin{eqnarray}
&&V(r,R(\theta ^{\prime },\varphi ^{\prime })) =\left[ V(r,R(\theta ^{\prime },
\varphi ^{\prime }))\right] _{(\delta \beta _{2}=0,\gamma =0,\beta _{3}=0)}+\notag\\%
&&+\left[ R_0 \frac{\partial }{\partial R}V(r,R(\theta ^{\prime },\varphi ^{\prime
}))\right] _{(\delta \beta _{2}=0,\gamma =0,\beta _{3}=0)}  \notag \\
&&\times \Bigg\{\beta _{20}\left[ \frac{\delta \beta _{2}}{\beta _{20}}\cos
\gamma +\cos \gamma -1\right] Y_{20}(\theta ^{\prime })+\notag \\ 
&&(\beta _{20}+\delta \beta _{2})\frac{\sin \gamma }{\sqrt{2}}\left[ Y_{22}(\theta ^{\prime
},\varphi ^{\prime })+Y_{2-2}(\theta ^{\prime },\varphi ^{\prime })\right]
\label{eq:pot_expansion} \\
&&+\beta _{3}\left[ \cos \eta \, Y_{30}(\theta ^{\prime })+\frac{\sin \eta }{%
\sqrt{2}}\left[ Y_{32}(\theta ^{\prime },\varphi ^{\prime })+Y_{3-2}(\theta
^{\prime },\varphi ^{\prime })\right] \right] \Bigg\}  \notag \\
&&+\text{ (члены высшего порядка по деформациям)},  \notag
\end{eqnarray}%

где 
\begin{itemize}
\item
$\left[ V(r,R(\theta ^{\prime},\varphi ^{\prime }))\right] _{(\delta \beta
_{2}=0,\gamma =0,\beta _{3}=0)}\equiv V_{rot}(r,R_{axial}(\theta ^{\prime
})) $,
\item $\left[ R_0 \frac{\partial }{\partial R}V(r,R(\theta ^{\prime
},\varphi ^{\prime }))\right] _{(\delta \beta _{2}=0,\gamma =0,\beta
_{3}=0)}\equiv \left[ R_0 \frac{\partial }{\partial R}
V(r,R(\theta ^{\prime},\varphi ^{\prime }))\right] _{R=R_{axial}(\theta^{\prime })}$,
\end{itemize}

\begin{equation}
R_{axial}(\theta ^{\prime })=R_{0}\left\{ 1+\underset{\lambda =2,4,6,8...}{%
\sum }\beta _{\lambda 0}Y_{\lambda 0}(\theta ^{\prime })\right\}.
\label{eq:raxial}
\end{equation}%

\begin{itemize}
\item
$V_{1}(r,\theta ^{\prime })\equiv
V_{rot}(r,R_{axial}(\theta ^{\prime }))=\left[ V(r,R(\theta ^{\prime
},\varphi ^{\prime }))\right] _{R=R_{axial}(\theta ^{\prime })}$
\item
$V_{2}(r,\theta ^{\prime })\equiv \left[  R_0 \frac{\partial }{\partial R}%
V(r,R(\theta ^{\prime },\varphi ^{\prime }))\right] _{R=R_{axial}(\theta
^{\prime })}$,
\end{itemize}
Мультипольное разложение величин $V_{i}(r,\theta
^{\prime })$ (with $i=1,2$) по аксиальным сферическим гармоникам $Y_{\lambda 0}(\theta ^{\prime })$ is performed:
\begin{equation}
V_{i}(r,\theta ^{\prime })=\sum_{\lambda ~(even)}v_{\lambda
}^{(i)}(r)Y_{\lambda 0}(\theta ^{\prime }),  \label{multipole1}
\end{equation}
где
\begin{equation}
v_{\lambda }^{(i)}(r)=2\pi \int_{0}^{\pi }V_{i}(r,\theta ^{\prime
})Y_{\lambda 0}(\theta ^{\prime })sin\theta ^{\prime }d\theta ^{\prime }.
\label{eq:vlambda}
\end{equation}
При этом $v_{\lambda }^{(1)}(r)$ соответствуют коэффициентам разложения Лежандра для жесткого ротатора,
уже доступные в существующих расчетных кодах, реализующих метод связанных каналов.

Подстановка этих обозначений в формулу \eqref{eq:pot_expansion} дает:
\begin{eqnarray}
V(r,\theta ^{\prime },\varphi ^{\prime }) &\equiv &V(r,R(\theta ^{\prime
},\varphi ^{\prime }))=\sum_{\lambda =0,2,4,..}v_{\lambda
}^{(1)}(r)Y_{\lambda 0}(\theta ^{\prime })+\sum_{\lambda =0,2,4..}v_{\lambda
}^{(2)}(r)Y_{\lambda 0}(\theta ^{\prime })\times  \notag \\
&\Bigg \{\beta _{20}&\left[ \frac{\delta \beta _{2}}{\beta _{20}}\cos \gamma
+\cos \gamma -1\right] Y_{20}(\theta ^{\prime })  \label{potential_expanded}
\\
&+&(\beta _{20}+\delta \beta _{2})\frac{\sin \gamma }{\sqrt{2}}\left[
Y_{22}(\theta ^{\prime },\varphi ^{\prime })+Y_{2-2}(\theta ^{\prime
},\varphi ^{\prime })\right]  \notag \\
&+&\beta _{3}\left[ \cos \eta \, Y_{30}(\theta ^{\prime })+\frac{\sin \eta }{%
\sqrt{2}}\left[ Y_{32}(\theta ^{\prime },\varphi ^{\prime })+Y_{3-2}(\theta
^{\prime },\varphi ^{\prime })\right] \right] \Bigg \} . \notag
\end{eqnarray}

By using the formula for the product of two spherical harmonics and
replacing $Y_{\lambda \nu }(\theta ^{\prime },\varphi ^{\prime })$
(expressed in the intrinsic system of reference) by $\sum_{\mu }D_{\mu ,\nu
}^{\lambda }Y_{\lambda \mu }(\theta ,\varphi )$ where $\theta $ and $\varphi
$ are the polar angles referred to the space-fixed coordinates and $D_{\mu
,\nu }^{\lambda }$ are the rotation Wigner functions, one obtains:
\begin{eqnarray}
V(r,\theta ,\varphi )&&=\sum_{\lambda =0,2,4,..}v_{\lambda
}^{(1)}(r)\sum_{\mu }D_{\mu ,0}^{\lambda }Y_{\lambda \mu }(\theta ,\varphi )
\label{V_multip_exp} \\
&&+\beta _{20}\left[ \frac{\delta \beta _{2}}{\beta _{20}}\cos \gamma +\cos
\gamma -1\right] \sum_{\lambda =0,2,4,..}\left[ \tilde{v}_{\lambda }^{(2)}(r)%
\right] _{0}\frac{1}{\sqrt{2}}\sum_{\mu }D_{\mu ,0}^{\lambda }Y_{\lambda \mu
}(\theta ,\varphi )  \notag \\
&&+(\beta _{20}+\delta \beta _{2})\frac{\sin \gamma }{\sqrt{2}}\sum_{\lambda
=2,4,6,..}\left[ \tilde{v}_{\lambda }^{(2)}(r)\right] _{2}\sum_{\mu }\left[
(D_{\mu ,2}^{\lambda }+D_{\mu ,-2}^{\lambda })Y_{\lambda \mu }(\theta
,\varphi )\right]  \notag \\
&&+\beta _{3}\left[ \cos \eta \, \sum_{\lambda =1,3,5,..}\left[ \tilde{v}%
_{\lambda }^{(3)}(r)\right] _{0}\sum_{\mu }D_{\mu ,0}^{\lambda }Y_{\lambda
\mu }(\theta ,\varphi )\right] \notag \\
&&+\beta _{3}\left[ \frac{\sin \eta }{\sqrt{2}}\sum_{\lambda =3,5,7,..}%
\left[ \tilde{v}_{\lambda }^{(3)}(r)\right] _{2}\sum_{\mu }\left[ (D_{\mu
,2}^{\lambda }+D_{\mu ,-2}^{\lambda })Y_{\lambda \mu }(\theta ,\varphi )%
\right] \right]  \notag \\
&&=V_{diag}(r)+V_{coupling}(r,\theta ,\varphi ),  \notag
\end{eqnarray}%
where $V_{diag}$ is the $\lambda =\mu =0$ component of the first term of Eq.~%
\eqref{V_multip_exp}, while $V_{coupling}$ is the rest and :\newline
\begin{eqnarray}
\left[ \tilde{v}_{\lambda }^{(2)}(r)\right] _{0} &=&\sum_{\lambda ^{\prime
}=0,2,4,..}v_{\lambda ^{\prime }}^{(2)}(r)\left[ \frac{5(2\lambda ^{\prime
}+1)}{4\pi (2\lambda +1)}\right] ^{1/2}\left\langle \lambda ^{\prime }200\vert\lambda 0\right\rangle ^{2},
\label{v20} \\
\left[ \tilde{v}_{\lambda }^{(2)}(r)\right] _{2} &=&\sum_{\lambda ^{\prime
}=0,2,4,...}v_{\lambda ^{\prime }}^{(2)}(r)\left[ \frac{5(2\lambda ^{\prime
}+1)}{4\pi (2\lambda +1)}\right] ^{1/2}\left\langle \lambda ^{\prime }200\vert\lambda
0\right\rangle \left\langle \lambda ^{\prime }202\vert\lambda 2\right\rangle ,  \label{v22} \\
\left[ \tilde{v}_{\lambda }^{(3)}(r)\right] _{0} &=&\sum_{\lambda ^{\prime
}=0,2,4,..}v_{\lambda ^{\prime }}^{(2)}(r)\left[ \frac{7(2\lambda ^{\prime
}+1)}{4\pi (2\lambda +1)}\right] ^{1/2}\left\langle \lambda ^{\prime }300\vert\lambda 0\right\rangle ^{2},
\label{v30} \\
\left[ \tilde{v}_{\lambda }^{(3)}(r)\right] _{2} &=&\sum_{\lambda ^{\prime
}=0,2,4,..}v_{\lambda ^{\prime }}^{(2)}(r)\left[ \frac{7(2\lambda ^{\prime
}+1)}{4\pi (2\lambda +1)}\right] ^{1/2}\left\langle \lambda ^{\prime }300\vert\lambda
0\right\rangle \left\langle \lambda ^{\prime }302\vert\lambda 2\right\rangle ,  \label{v32}
\end{eqnarray}%
with the following constraints:

\begin{itemize}
	\item[-] Only {\bfseries even $\lambda ^{\prime }$} values are allowed in
	the summations which appear in the expressions of $\tilde{v}_{\lambda
	}^{(2)}(r)$ and $\tilde{v}_{\lambda }^{(3)}(r)$ in Eqs.~\eqref{v20} to %
	\eqref{v32}, since $v_{\lambda ^{\prime }}^{(2)}=0$ for odd $\lambda
	^{\prime }$ (from the axial and reflection symmetry of the equilibrium
	nuclear shape).
	
	\item[-] Only {\bfseries even $\lambda $} values are allowed for the
	quadrupole vibrational coupling $\tilde{v}_{\lambda }^{(2)}(r)$ because of
	the\ $\left\langle \lambda ^{\prime }200\vert\lambda 0\right\rangle $ Clebsch-Gordan coefficient in Eqs.~%
	\eqref{v20} and \eqref{v22}, since $\lambda ^{\prime }$ is even.
	
	\item[-] Only {\bfseries odd $\lambda $} values are allowed for the octupole
	vibrational coupling term $\tilde{v}_{\lambda }^{(3)}(r)$ because of the $%
	\left\langle \lambda ^{\prime }300\vert\lambda 0\right\rangle $ Clebsch-Gordan coefficient in Eqs.~%
	\eqref{v30} and \eqref{v32}, since $\lambda ^{\prime }$ is even.
	
	\item[-] $[\tilde{v}_{0}^{(2)}]_{2}=[\tilde{v}_{1}^{(3)}]_{2}=0$ due to the
	second Clebsch-Gordan coefficient in Eqs.~\eqref{v22} and \eqref{v32}, as
	projection $\mu =2$ can not be larger than corresponding momentum $\lambda $.
\end{itemize}

The deformed optical model potential of Eq.~\eqref{V_multip_exp}, through
its variables $\delta \beta _{2}$, $\gamma $, $\beta _{3}$, and $\eta$
($\beta _{20}$ is the static deformation and doesn't play any role in the
couplings) guarantees coupling of vibrational bands (coupling between
rotational states, described by the Wigner's $D$-funtions, are made
through the spherical harmonics). Thus, for example, the first term usually corresponds
to the intra-band coupling (e.g., couples the rotational band members built
on the same vibrational state); the second term couples the ground-state band with quadrupolar vibrational
states (e.g., $\beta $-vibrational and $\gamma $-vibrational bands); the third term couples states
in the ground-state band with states of the octupolar vibrational band;  and so on.


\section{Матричные элементы оптического потенциала для расширенной схемы связи}

\section{Учет сохранения объема ядра при осцилляциях формы в рамках модели
мягкого ротатора}

\section{Насыщение схемы связанных уровней на примере ядра 238U}

\section{Возможные подходы к описанию нечетных ядер в рамках предлагаемой 
оптической модели}
