\pdfbookmark{Общая характеристика работы}{characteristic}             % Закладка pdf
\section*{Общая характеристика работы}

\newcommand{\actuality}{\pdfbookmark[1]{Актуальность}{actuality}\underline{\textbf{\actualityTXT}}}
\newcommand{\progress}{\pdfbookmark[1]{Разработанность темы}{progress}\underline{\textbf{\progressTXT}}}
\newcommand{\aim}{\pdfbookmark[1]{Цели}{aim}\underline{{\textbf\aimTXT}}}
\newcommand{\tasks}{\pdfbookmark[1]{Задачи}{tasks}\underline{\textbf{\tasksTXT}}}
\newcommand{\aimtasks}{\pdfbookmark[1]{Цели и задачи}{aimtasks}\aimtasksTXT}
\newcommand{\novelty}{\pdfbookmark[1]{Научная новизна}{novelty}\underline{\textbf{\noveltyTXT}}}
\newcommand{\influence}{\pdfbookmark[1]{Практическая значимость}{influence}\underline{\textbf{\influenceTXT}}}
\newcommand{\methods}{\pdfbookmark[1]{Методология и методы исследования}{methods}\underline{\textbf{\methodsTXT}}}
\newcommand{\defpositions}{\pdfbookmark[1]{Положения, выносимые на защиту}{defpositions}\underline{\textbf{\defpositionsTXT}}}
\newcommand{\reliability}{\pdfbookmark[1]{Достоверность}{reliability}\underline{\textbf{\reliabilityTXT}}}
\newcommand{\probation}{\pdfbookmark[1]{Апробация}{probation}\underline{\textbf{\probationTXT}}}
\newcommand{\contribution}{\pdfbookmark[1]{Личный вклад}{contribution}\underline{\textbf{\contributionTXT}}}
\newcommand{\publications}{\pdfbookmark[1]{Публикации}{publications}\underline{\textbf{\publicationsTXT}}}

Обзор, введение в тему, обозначение места данной работы в
мировых исследованиях и~т.\:п., можно использовать ссылки на~другие
работы~\autocite{Gosele1999161,Lermontov}
(если их~нет, то~в~автореферате
автоматически пропадёт раздел <<Список литературы>>). Внимание! Ссылки
на~другие работы в~разделе общей характеристики работы можно
использовать только при использовании \verb!biblatex! (из-за технических
ограничений \verb!bibtex8!. Это связано с тем, что одна
и~та~же~характеристика используются и~в~тексте диссертации, и в
автореферате. В~последнем, согласно ГОСТ, должен присутствовать список
работ автора по~теме диссертации, а~\verb!bibtex8! не~умеет выводить в~одном
файле два списка литературы).
При использовании \verb!biblatex! возможно использование исключительно
в~автореферате подстрочных ссылок
для других работ командой \verb!\autocite!, а~также цитирование
собственных работ командой \verb!\cite!. Для этого в~файле
\verb!common/setup.tex! необходимо присвоить положительное значение
счётчику \verb!\setcounter{usefootcite}{1}!.

Для генерации содержимого титульного листа автореферата, диссертации
и~презентации используются данные из файла \verb!common/data.tex!. Если,
например, вы меняете название диссертации, то оно автоматически
появится в~итоговых файлах после очередного запуска \LaTeX. Согласно
ГОСТ 7.0.11-2011 <<5.1.1 Титульный лист является первой страницей
диссертации, служит источником информации, необходимой для обработки и
поиска документа>>. Наличие логотипа организации на~титульном листе
упрощает обработку и~поиск, для этого разметите логотип вашей
организации в папке images в~формате PDF (лучше найти его в векторном
варианте, чтобы он хорошо смотрелся при печати) под именем
\verb!logo.pdf!. Настроить размер изображения с логотипом можно
в~соответствующих местах файлов \verb!title.tex!  отдельно для
диссертации и автореферата. Если вам логотип не~нужен, то просто
удалите файл с~логотипом.

\ifsynopsis
Этот абзац появляется только в~автореферате.
Для формирования блоков, которые будут обрабатываться только в~автореферате,
заведена проверка условия \verb!\!\verb!ifsynopsis!.
Значение условия задаётся в~основном файле документа (\verb!synopsis.tex! для
автореферата).
\else
Этот абзац появляется только в~диссертации.
Через проверку условия \verb!\!\verb!ifsynopsis!, задаваемого в~основном файле
документа (\verb!dissertation.tex! для диссертации), можно сделать новую
команду, обеспечивающую появление цитаты в~диссертации, но~не~в~автореферате.
\fi

% {\progress}
% Этот раздел должен быть отдельным структурным элементом по
% ГОСТ, но он, как правило, включается в описание актуальности
% темы. Нужен он отдельным структурынм элемементом или нет ---
% смотрите другие диссертации вашего совета, скорее всего не нужен.


\subsection*{Cвязь работы с научными программами (проектами), темами}

Исследования, результаты которых вошли в диссертацию, проводились в {\thesisInOrganization} в рамках следующих государственных программ и проектов:

\begin{itemize}
	\item ГПНИ <<Атомная энергетика, ядерные и радиационные технологии>>, задание 3.1.17 «Получение оптических потенциалов, основанных на современных моделях связи каналов, для расчета с гарантированной точностью сечений нуклон-ядерных взаимодействий», №~гос.~рег. 20141673, 2014--2015 гг.
	\item ГПНИ <<Энергетические системы, процессы и технологии>>, задание 3.1.01 <<Разработка алгоритмов распараллеливания вычислений и внедрение их в программный комплекс OPTMAN для существенного (в десятки раз) ускорения расчетов сечений взаимодействия нуклонов с ядрами, необходимых для константного обеспечения задач, важных для народного хозяйства Беларуси>>, №~гос.~рег. 20160605, 2016--2020 гг.
	\item ГПНИ <<Энергетические и ядерные процессы и технологии>>, задание 3.1.03.2 «Разработка современных теоретических моделей и программных комплексов расчета сечений взаимодействия с ядрами для предсказания нуклонных сечений с гарантированной точностью», №~гос.~рег. 20210866, 2021--2025 гг.
	\item Исследовательский контракт с МАГАТЭ №~19263 <<Coupled-Channel Optical Model Potential for Even-Even Minor Actinides Using Extended Couplings>>, 2017--2021 гг.
\end{itemize}

Тема диссертации соответствует приоритетным направлениям научно-технической деятельности в Республике Беларусь, в частности, пункту 1 <<Энергетика и энергоэффективность, атомная энергетика: атомная энергетика>> перечня приоритетных направлений научно-технической деятельности в Республике Беларусь на 2016--2020 годы, утвержденного указом Президента Республики Беларусь от 22~апреля 2015~г. №~166, а также пункту 3 <<Энергетика, строительство, экология и рациональное природопользование: атомная энергетика, ядерная и радиационная безопасность>> перечня приоритетных направлений научно-технической деятельности в Республике Беларусь на 2021--2025 годы, утвержденного указом Президента Республики Беларусь от 7~мая 2020~г. №~156.

Диссертация соответствует пунктам <<Характеристики, свойства и энергетические спектры ядер. Теория структуры и модели ядра. Деление и синтез ядер, ядерная нейтринная физика>> и <<Ядерные реакции и теория рассеяния. Релятивистская ядерная физика и физика тяжелых ионов>> паспорта специальности <<\thesisSpecialtyNumber~--- \thesisSpecialtyTitle>>.


\subsection*{Цель и задачи исследования}

\textit{Целью} данной работы является описание оптических сечений рассеяния нуклонов на четно-четных и нечетных актинидах с учетом мягкости и деформаций ядер.

Для~достижения поставленной цели необходимо было решить следующие \textit{задачи}:
\begin{enumerate}[beginpenalty=10000] % https://tex.stackexchange.com/a/476052/104425
	\item Развить оптическую модель для учета как колебательных, так и вращательных возбуждений в мягких деформированных ядрах
	\item Вычислить матричные элементы операторов деформаций для четно-четных и нечетных актинидов
	\item Получить региональный оптический потенциал для описания рассеяния нуклонов на актинидах
	\item Проанализировать изменения в предсказываемых сечениях при использовании новой модели
\end{enumerate}


\textit{Объект исследования}: реакции рассеяния нуклонов на тяжелых деформируемых ядрах с большой равновесной квадрупольной деформацией


\textit{Предмет исследования}: оптическая модель ядра в рамках метода сильной связи каналов, учитывающая колебательные и вращательные возбуждения ядра-мишени.

\subsection*{Научная новизна}

%{\novelty}
\begin{enumerate}[beginpenalty=10000] % https://tex.stackexchange.com/a/476052/104425
	\item Впервые был при
	\item Впервые \ldots
	\item Было выполнено оригинальное исследование \ldots
\end{enumerate}

\subsection*{Положения, выносимые на защиту}

%{\defpositions}
\begin{enumerate}[beginpenalty=10000] % https://tex.stackexchange.com/a/476052/104425
	\item Дополнительные члены в матричном элементе связи каналов, возникающие при рассмотрении мягкости деформированного ядра, определяются произведением следующих сомножителей: квадрупольным моментом оптического потенциала, матричным элементом оператора деформации в базисе состояний ядра и функцией угловых моментов канала рассеяния.
	\item Матричные элементы операторов деформаций (эффективные деформации) для нечетных ядер могут быть приближенно оценены как эффективные деформации ядра, соответствующего четно-четному остову в рамках модели мягкого ротатора.
	\item Учет мягкости деформированного ядра при рассмотрении анализе нейтронов в рамках оптической модели приводит к рассмотрению 3 дополнительных эффектов: связь с уровнями нескольких вращательных полос, растяжение ядра при вращении, дополнительная связь каналов за счет коррекции сохранения объема ядра при колебаниях поверхности.
	\item Изменения в сечении образования составного ядра по сравнения с предсказаниями в рамках модели жесткого ротатора достигают 10\% для четно-четных и нечетных актинидов в диапазоне энергий налетающих нейтронов 0,1--10~МэВ.
\end{enumerate}
В папке Documents можно ознакомиться с решением совета из Томского~ГУ
(в~файле \verb+Def_positions.pdf+), где обоснованно даются рекомендации
по~формулировкам защищаемых положений.

\subsection*{Личный вклад соискателя ученой степени}

%{\contribution} 
Автор принимал активное участие \ldots

\subsection*{Апробация диссертации и информация об использовании ее результатов}

%{\probation}
Основные результаты диссертации докладывались~на международных конференциях:

\begin{enumerate}
	\item LXV Международная конференция по ядерной физике <<Новые горизонты в области ядерной физики, атомной, фемто- и нанотехнологий>> (Ядро-2015), Санкт-Петербург, Россия, 29 июня--3 июля 2015 года;
	
	\item Конференция молодых специалистов <<Инновации в атомной энергетике>>, НИКИЭТ им. Н.\,А.\,Доллежаля, Москва, Россия, 25--26 ноября 2015;
	
	\item <<International conference on nuclear data for science and technology>> (ND2016), Брюгге, Бельгия, 11--16 сентября 2016 года;
	
	\item VI Международная конференция <<Ядерные технологии XXI века>>, Минск,  Беларусь, 25--28 октября 2016 года;
	
	\item <<Joint ICTP-IAEA Workshop on the Evaluation of Nuclear Reaction Data for Applications>>, Триест, Италия, 2-13 октября 2017 года;
	
	\item <<5th International Workshop On Nuclear Data Evaluation for Reactor Applications>> (WONDER-2018), Экс-ан-Прованс, Франция, 8--12 октября 2018 года;
	
	\item <<International Conference on Nuclear Data for Science and Technology>> (ND2019), Пекин, Китай, 19--24 мая 2019 года;
	
	\item <<XXVIII International Seminar Nonlinear Phenomena in Complex Systems>>, Минск, Беларусь, 18--21 мая 2021 года;
	
	\item <<28 International Seminar on Interaction of Neutrons with Nuclei>> (ISINN-28), Дубна, Россия, 24--28 мая 2021 года.
\end{enumerate}


\subsection*{Опубликование результатов диссертации}

\ifnumequal{\value{bibliosel}}{0}
{%%% Встроенная реализация с загрузкой файла через движок bibtex8. (При желании, внутри можно использовать обычные ссылки, наподобие `\cite{vakbib1,vakbib2}`).
	{\publications} Основные результаты по теме диссертации изложены
	в~XX~печатных изданиях,
	X из которых изданы в журналах, рекомендованных ВАК,
	X "--- в тезисах докладов.
}%
{%%% Реализация пакетом biblatex через движок biber
	\begin{refsection}[bl-author, bl-registered]
		% Это refsection=1.
		% Процитированные здесь работы:
		%  * подсчитываются, для автоматического составления фразы "Основные результаты ..."
		%  * попадают в авторскую библиографию, при usefootcite==0 и стиле `\insertbiblioauthor` или `\insertbiblioauthorgrouped`
		%  * нумеруются там в зависимости от порядка команд `\printbibliography` в этом разделе.
		%  * при использовании `\insertbiblioauthorgrouped`, порядок команд `\printbibliography` в нём должен быть тем же (см. biblio/biblatex.tex)
		%
		% Невидимый библиографический список для подсчёта количества публикаций:
		\printbibliography[heading=nobibheading, section=1, env=countauthorvak,          keyword=biblioauthorvak]%
		\printbibliography[heading=nobibheading, section=1, env=countauthorwos,          keyword=biblioauthorwos]%
		\printbibliography[heading=nobibheading, section=1, env=countauthorscopus,       keyword=biblioauthorscopus]%
		\printbibliography[heading=nobibheading, section=1, env=countauthorconf,         keyword=biblioauthorconf]%		
		\printbibliography[heading=nobibheading, section=1, env=countauthorconfshort,         keyword=biblioauthorconfshort]%
		\printbibliography[heading=nobibheading, section=1, env=countauthorother,        keyword=biblioauthorother]%
		\printbibliography[heading=nobibheading, section=1, env=countregistered,         keyword=biblioregistered]%
		\printbibliography[heading=nobibheading, section=1, env=countauthorpatent,       keyword=biblioauthorpatent]%
		\printbibliography[heading=nobibheading, section=1, env=countauthorprogram,      keyword=biblioauthorprogram]%
		\printbibliography[heading=nobibheading, section=1, env=countauthor,             keyword=biblioauthor]%
		\printbibliography[heading=nobibheading, section=1, env=countauthorvakscopuswos, filter=vakscopuswos]%
		\printbibliography[heading=nobibheading, section=1, env=countauthorscopuswos,    filter=scopuswos]%
		%
		\nocite{*}%
		%
		%{\publications} 
		Основные результаты по теме диссертации изложены в~\arabic{citeauthor}~печатных изданиях,
		\arabic{citeauthorvak} из которых изданы в журналах, рекомендованных ВАК\sloppy%
		\ifnum \value{citeauthorscopuswos}>0%
		, \arabic{citeauthorscopuswos} "--- в~периодических научных журналах, индексируемых Web of~Science и Scopus\sloppy%
		\fi%
		\ifnum \value{citeauthorconf}>0%
		, \arabic{citeauthorconf} "--- статьи в сборниках материалов конференций%
		\fi%
		\ifnum \value{citeauthorconfshort}>0%
		, \arabic{citeauthorconfshort} "--- в~тезисах докладов.
		\else%
		.
		\fi%
		\ifnum \value{citeregistered}=1%
		\ifnum \value{citeauthorpatent}=1%
		Зарегистрирован \arabic{citeauthorpatent} патент.
		\fi%
		\ifnum \value{citeauthorprogram}=1%
		Зарегистрирована \arabic{citeauthorprogram} программа для ЭВМ.
		\fi%
		\fi%
		\ifnum \value{citeregistered}>1%
		Зарегистрированы\ %
		\ifnum \value{citeauthorpatent}>0%
		\formbytotal{citeauthorpatent}{патент}{}{а}{}\sloppy%
		\ifnum \value{citeauthorprogram}=0 . \else \ и~\fi%
		\fi%
		\ifnum \value{citeauthorprogram}>0%
		\formbytotal{citeauthorprogram}{программ}{а}{ы}{} для ЭВМ.
		\fi%
		\fi%
		% К публикациям, в которых излагаются основные научные результаты диссертации на соискание учёной
		% степени, в рецензируемых изданиях приравниваются патенты на изобретения, патенты (свидетельства) на
		% полезную модель, патенты на промышленный образец, патенты на селекционные достижения, свидетельства
		% на программу для электронных вычислительных машин, базу данных, топологию интегральных микросхем,
		% зарегистрированные в установленном порядке.(в ред. Постановления Правительства РФ от 21.04.2016 N 335)
	\end{refsection}%
	\begin{refsection}[bl-author, bl-registered]
		% Это refsection=2.
		% Процитированные здесь работы:
		%  * попадают в авторскую библиографию, при usefootcite==0 и стиле `\insertbiblioauthorimportant`.
		%  * ни на что не влияют в противном случае
		\nocite{*}
		%        \nocite{vakbib2}%vak
		%        \nocite{patbib1}%patent
		%        \nocite{progbib1}%program
		%        \nocite{bib1}%other
		%        \nocite{confbib1}%conf
	\end{refsection}%
	%
	% Всё, что вне этих двух refsection, это refsection=0,
	%  * для диссертации - это нормальные ссылки, попадающие в обычную библиографию
	%  * для автореферата:
	%     * при usefootcite==0, ссылка корректно сработает только для источника из `external.bib`. Для своих работ --- напечатает "[0]" (и даже Warning не вылезет).
	%     * при usefootcite==1, ссылка сработает нормально. В авторской библиографии будут только процитированные в refsection=0 работы.
}

При использовании пакета \verb!biblatex! будут подсчитаны все работы, добавленные
в файл \verb!biblio/author.bib!. Для правильного подсчёта работ в~различных
системах цитирования требуется использовать поля:
\begin{itemize}
	\item \texttt{authorvak} если публикация индексирована ВАК,
	\item \texttt{authorscopus} если публикация индексирована Scopus,
	\item \texttt{authorwos} если публикация индексирована Web of Science,
	\item \texttt{authorconfshort} для тезисов конференций,
	\item \texttt{authorconf} для статей в сборниках материалов конференций,
	\item \texttt{authorpatent} для патентов,
	\item \texttt{authorprogram} для зарегистрированных программ для ЭВМ,
	\item \texttt{authorother} для других публикаций.
\end{itemize}
Для подсчёта используются счётчики:
\begin{itemize}
	\item \texttt{citeauthorvak} для работ, индексируемых ВАК,
	\item \texttt{citeauthorscopus} для работ, индексируемых Scopus,
	\item \texttt{citeauthorwos} для работ, индексируемых Web of Science,
	\item \texttt{citeauthorvakscopuswos} для работ, индексируемых одной из трёх баз,
	\item \texttt{citeauthorscopuswos} для работ, индексируемых Scopus или Web of~Science,
	\item \texttt{authorconfshort} для тезисов конференций,
	\item \texttt{authorconf} для статей в сборниках материалов конференций,
	\item \texttt{citeauthorother} для остальных работ,
	\item \texttt{citeauthorpatent} для патентов,
	\item \texttt{citeauthorprogram} для зарегистрированных программ для ЭВМ,
	\item \texttt{citeauthor} для суммарного количества работ.
\end{itemize}
% Счётчик \texttt{citeexternal} используется для подсчёта процитированных публикаций;
% \texttt{citeregistered} "--- для подсчёта суммарного количества патентов и программ для ЭВМ.

Для добавления в список публикаций автора работ, которые не были процитированы в
автореферате, требуется их~перечислить с использованием команды \verb!\nocite! в
\verb!Synopsis/content.tex!.

\subsection*{Структура и объем диссертации}


Диссертационная работа состоит из \todo{перечня условных обозначений}, введения, общей характеристики работы, четырех глав, заключения, библиографического списка и \todo{приложения}. Полный объем диссертации составляет \todo{ХХХ} страниц, работа содержит \todo{ХХХ} рисунков, \todo{ХХХ} таблиц. Библиографический список состоит их \todo{ХХХ} наименований.


Диссертация состоит из~введения,
%\formbytotal{totalchapter}{глав}{ы}{}{},
заключения и
%\formbytotal{totalappendix}{приложен}{ия}{ий}{}.
%% на случай ошибок оставляю исходный кусок на месте, закомментированным
%Полный объём диссертации составляет  \ref*{TotPages}~страницу
%с~\totalfigures{}~рисунками и~\totaltables{}~таблицами. Список литературы
%содержит \total{citenum}~наименований.
%
Полный объём диссертации составляет
%\formbytotal{TotPages}{страниц}{у}{ы}{}, включая
%\formbytotal{totalcount@figure}{рисун}{ок}{ка}{ков} и
%\formbytotal{totalcount@table}{таблиц}{у}{ы}{}.
Список литературы содержит
%\formbytotal{citenum}{наименован}{ие}{ия}{ий}.



%{\influence} \ldots

%{\methods} \ldots


%{\reliability} полученных результатов обеспечивается \ldots \ Результаты находятся в соответствии с результатами, полученными другими авторами.





 % Характеристика работы по структуре во введении и в автореферате не отличается (ГОСТ Р 7.0.11, пункты 5.3.1 и 9.2.1), потому её загружаем из одного и того же внешнего файла, предварительно задав форму выделения некоторым параметрам

%Диссертационная работа была выполнена при поддержке грантов \dots

%\underline{\textbf{Объем и структура работы.}} Диссертация состоит из~введения,
%четырех глав, заключения и~приложения. Полный объем диссертации
%\textbf{ХХХ}~страниц текста с~\textbf{ХХ}~рисунками и~5~таблицами. Список
%литературы содержит \textbf{ХХX}~наименование.

\pdfbookmark{Основная часть}{description}                          % Закладка pdf
\section*{Основная часть}
Во \underline{\textbf{введении}} обосновывается актуальность
исследований, проводимых в~рамках данной диссертационной работы,
приводится обзор научной литературы по~изучаемой проблеме,
формулируется цель, ставятся задачи работы, излагается научная новизна
и практическая значимость представляемой работы. В~последующих главах
сначала описывается общий принцип, позволяющий \dots, а~потом идёт
апробация на частных примерах: \dots  и~\dots.


\underline{\textbf{Первая глава}} посвящена \dots

картинку можно добавить так:
\begin{figure}[ht]
    \centerfloat{
        \hfill
        \subcaptionbox{\LaTeX}{%
            \includegraphics[scale=0.27]{latex}}
        \hfill
        \subcaptionbox{Knuth}{%
            \includegraphics[width=0.25\linewidth]{knuth1}}
        \hfill
    }
    \caption{Подпись к картинке.}\label{fig:latex}
\end{figure}

Формулы в строку без номера добавляются так:
\[
    \lambda_{T_s} = K_x\frac{d{x}}{d{T_s}}, \qquad
    \lambda_{q_s} = K_x\frac{d{x}}{d{q_s}},
\]

\underline{\textbf{Вторая глава}} посвящена исследованию

\underline{\textbf{Третья глава}} посвящена исследованию

Можно сослаться на свои работы в автореферате. Для этого в файле
\verb!Synopsis/setup.tex! необходимо присвоить положительное значение
счётчику \verb!\setcounter{usefootcite}{1}!. В таком случае ссылки на
работы других авторов будут подстрочными.
Изложенные в третьей главе результаты опубликованы в~\cite{Quesada2017, Martyanov2017a}.
Использование подстрочных ссылок внутри таблиц может вызывать проблемы.

В \underline{\textbf{четвертой главе}} приведено описание

\FloatBarrier


\pdfbookmark{Заключение}{conclusion}                          % Закладка pdf
\section*{Заключение}


%% Согласно ГОСТ Р 7.0.11-2011:
%% 5.3.3 В заключении диссертации излагают итоги выполненного исследования, рекомендации, перспективы дальнейшей разработки темы.
%% 9.2.3 В заключении автореферата диссертации излагают итоги данного исследования, рекомендации и перспективы дальнейшей разработки темы.

\pdfbookmark[1]{Основные научные результаты диссертации}{results}   
\subsection*{Основные научные результаты диссертации}

\begin{enumerate}
  \item На основе анализа \ldots
  \item Численные исследования показали, что \ldots
  \item Математическое моделирование показало \ldots
  \item Для выполнения поставленных задач был создан \ldots
\end{enumerate}

\pdfbookmark[1]{Рекомендации по практическому использованию результатов}{practice}   
\subsection*{Рекомендации по практическому использованию результатов}

Ставь в печку на 2 часа - и готово!

\pdfbookmark{Литература}{bibliography}                                % Закладка pdf
При использовании пакета \verb!biblatex! список публикаций автора по теме
диссертации формируется в разделе <<\publications>>\ файла
\verb!common/characteristic.tex!  при помощи команды \verb!\nocite!

\ifdefmacro{\microtypesetup}{\microtypesetup{protrusion=false}}{} % не рекомендуется применять пакет микротипографики к автоматически генерируемому списку литературы
\urlstyle{rm}                               % ссылки URL обычным шрифтом
\ifnumequal{\value{bibliosel}}{0}{% Встроенная реализация с загрузкой файла через движок bibtex8
    \renewcommand{\bibname}{\large \bibtitleauthor}
    \nocite{*}
    \insertbiblioauthor           % Подключаем Bib-базы
    %\insertbiblioexternal   % !!! bibtex не умеет работать с несколькими библиографиями !!!
}{% Реализация пакетом biblatex через движок biber
    % Цитирования.
    %  * Порядок перечисления определяет порядок в библиографии (только внутри подраздела, если `\insertbiblioauthorgrouped`).
    %  * Если не соблюдать порядок "как для \printbibliography", нумерация в `\insertbiblioauthor` будет кривой.
    %  * Если цитировать каждый источник отдельной командой --- найти некоторые ошибки будет проще.
    %
    \nocite{*}
%    %% authorvak
%    vakbib2{vakbib1}%
%    vakbib2{vakbib2}%
%    %
%    %% authorwos
%    vakbib2{wosbib1}%
%    %
%    %% authorscopus
%    \nocite{scbib1}%
%    %
%    %% authorpathent
%    \nocite{patbib1}%
%    %
%    %% authorprogram
%    \nocite{progbib1}%
%    %
%    %% authorconf
%    \nocite{confbib1}%
%    \nocite{confbib2}%
%    %
%    %% authorother
%    \nocite{bib1}%
%    \nocite{bib2}%

    \ifnumgreater{\value{usefootcite}}{0}{
        \begin{refcontext}[labelprefix={}]
            \ifnum \value{bibgrouped}>0
                \insertbiblioauthorgrouped    % Вывод всех работ автора, сгруппированных по источникам
            \else
                \insertbiblioauthor      % Вывод всех работ автора
            \fi
        \end{refcontext}
    }{
        \ifnum \totvalue{citeexternal}>0
            \begin{refcontext}[labelprefix=A]
                \ifnum \value{bibgrouped}>0
                    \insertbiblioauthorgrouped    % Вывод всех работ автора, сгруппированных по источникам
                \else
                    \insertbiblioauthor      % Вывод всех работ автора
                \fi
            \end{refcontext}
        \else
            \ifnum \value{bibgrouped}>0
                \insertbiblioauthorgrouped    % Вывод всех работ автора, сгруппированных по источникам
            \else
                \insertbiblioauthor      % Вывод всех работ автора
            \fi
        \fi
        %  \insertbiblioauthorimportant  % Вывод наиболее значимых работ автора (определяется в файле characteristic во второй section)
        \begin{refcontext}[labelprefix={}]
            \insertbiblioexternal            % Вывод списка литературы, на которую ссылались в тексте автореферата
        \end{refcontext}
        % Невидимый библиографический список для подсчёта количества внешних публикаций
        % Используется, чтобы убрать приставку "А" у работ автора, если в автореферате нет
        % цитирований внешних источников.
        \printbibliography[heading=nobibheading, section=0, env=countexternal, keyword=biblioexternal, resetnumbers=true]%
    }
}
\ifdefmacro{\microtypesetup}{\microtypesetup{protrusion=true}}{}
\urlstyle{tt}                               % возвращаем установки шрифта ссылок URL
